\subsection{Alex Pitcher Awards}

As part of the funds coordinated by the Ghar Parau Foundation, we received two Alex Pitcher
memorial fund awards for Myles Denton and Kate Smith, who were both first year
undergraduate students going on their first caving expedition.

\subsubsection{Myles Denton}

In July 2010, I was about to embark on my first caving expedition, at the end
of a year's caving in the UK.  It's easy to say that I was a little excited.

I have always been an adventurous person, I adore the outdoors and I have a
great interest in the Earth and thus joining ICCC in October was the inevitable
outcome!  I had never been caving before, and from the initial presentations
and meetings, I was extremely keen to get below-ground.  Soon enough, I was
hooked and was looking forward to heading off on a 4 week expedition to the
Julian Alps in Slovenia.

Slovenia was one of the hardest and most rewarding experiences I have had.
After packing up a month's worth of supplies and cramming as much of it into
the minibus as possible, our group of intrepid cavers set off towards Slovenia,
a journey taking us through six countries over twenty four hours.

Arriving in Slovenia I was immediately taken aback by the country's beauty, and
after spending a night eating pizza and sleeping in Tolmin, we began are ascent
of Migovec.  The mountain plateau where our base camp was set up is at an
altitude between 1800m and 2000m.  Driving to a farm at Ravne, we shared
coffee, packed any remaining space in our bags with food, and headed up
Migovec.  The first trip to the top was accompanied by a rain cloud, which left
us wet, but nicely cool up to the top.  The next few days were taken up by
cavers making trips up and down the mountain, ferrying equipment, food and most
importantly huge volumes of very heavy cheese!

After being showed around the plateau, introduced to the various cave entrances
and schooled on the history of the caving on the mountain Nikolas Kral and I
prepared for our first sortie underground, with experienced caver James
"Tetley" Hooper.  Our first trip was in the Vrtnarija cave
system.  Descending through the entrance dubbed Gardener's World,
the three of us made a steady descent to the bottom of the Pico pitch, at about
-200 m.  This was already deeper than either of us had been in our nine month
caving experience in the UK!  The following day, infested by the caving bug,
Tetley and I again entered Vrtnarija, this time descending to the top of
Fistful of Tolars pitch, at -400 m.  These trips also doubled up as
opportunities to take some of the gear needed to set up camp at -550 m.  Several
other caving trips were made, taking the nine tackle bags down to the middle of
Friendship Gallery.  These included Kate Smith, another fresher, on her first
expedition with Nick and I.

The cave system within Migovec is genuinely breath taking.  Whilst not the most
decorated of caves (which we later discover not to be true), the shear nature
of the cave is beautiful.  Descending through large chambers, with pitches up
to 120m, is an exhilarating experience, learning to slide through those tricky
parts of the cave and strolling through the large phreatic passages is
incredibly liberating.

Soon enough, I was ready to head to Camp X-Ray, our underground camp at -550 m.
Our underground camp consisted of a 4 man single layer tent, which helped to
raise the temperature of 1$^\circ$ C by a few degrees.  We had two Vango Nitestar 450 
sleeping bags, which are brilliant, and a set of two buffalo bags, which
enabled our camp to become a four man camp.  We used methylated spirits and gas
stoves to cook with, and had several candles for ambience!  As well as a first
aid kit, we had a small set of speakers, a small video system and an mp3 full
of Blackadder.  At -550 m, all this made underground camp a very welcome place
indeed.

On my first trip to underground camp Tetley and I spent one night there,
pushing the Tolminski Korita series for several hours, and discovering the
pitch later named Black Knight.  We were joined by Gergely and Andy 
after our stay at underground camp, who informed us they'd been pushing a messy
pitch named Leopard, which had burst into a decorative horizontal passage which
had many leads.  We were all excited.  On the surface, I helped input our
survey data into the computer, and reviewed the new passage we had discovered.  

Over the next week or so, several caving parties went below, mainly pushing the
Leopard series.  Tetley and I returned to push Black Knight, which entered
further pitches, and then hit a duck.  Here I was worried the passage was
blocked by the water, however, Tetley cunningly discovered a bypass passage
which passed over and around the duck.  Jana Carga and Jarvist Moore frost
pushed the series and closed on part, Sidewinder, as it dropped into a part of
the cave known as Envy, at about -660 m.  On my last pushing trip towards the
end of expedition, Jarvist and I closed the final part of the Black Knight
series, named Stalemate, where the water flowed into a narrow rift too small
for us to squeeze through despite our attempts.  

Overall, my Slovenian expedition was life changing.  2.2 km of undiscovered cave
was found, between about twenty cavers.  
Personally I discovered around 310m of passage with my partners.
More importantly, my caving skills improved hugely, I went from taking 3 hours
to get to - 550m to 1 hour 45 minutes, I learnt how to install bolts, how to tie
many knots, and where to use them.  My techniques for both abseiling and
prussiking improved greatly, and overall I became more confident and proficient
in all aspects of caving.  I don't think twice about hauling several tackle
bags down a cave anymore!  The funding from Alex Pitcher helped me to purchase
a Petzl helmet and Mig light attachment, which I used on every underground
trip.    

\subsubsection{Kate Smith}

From 15th July to 15th August 2010, I joined Imperial College Caving Club on
their summer expedition to the Julian Alps to explore the expansive Slovenian
cave systems there. ICCC have been present in the area almost annually since
1994, running joint exploration with a local club from the town of Tolmin, the
JSPDT. From Tolmin we travelled 6km to the large limestone plateau of Tolminski
Migovec, within the Triglav national park. Here I lived for 27 days with
various members of the expedition, unified by the sole motive to find and
survey virgin cave.

Six of us set off to Slovenia in a university minibus that was jam-packed with
2km of rope, food and caving equipment. 24 hours after leaving London and
passing through France, Belgium, Germany, Austria and Italy we arrived at a
member's flat in Tolmin. We were warmly welcomed by the members of JSPDT and
the ICCC members who had arrived some days before to set up camp on the
mountain. As one of three new cavers to the expedition (having been caving for
barely a year). I was met with much excitement and plans for my first caving
trip in the mountains.

The next morning we drove some way up the mountain to a farm belonging to a
family kind enough to lend us some barn space to keep our supplies and
equipment. From here we carried as much as we could the rest of the way up
Migovec to 1880m above sea level, each member making several 'carries' over the
course of the expedition.

Camp on Mig consists of tents dotted around the plateau and a communal living
space known as the Bivi. The Bivi is a nice big shake hole with a little cave
and an impressive stone bridge to provide some shelter. Here we cook, eat,
drink and occasionally take naps. Tarpaulin is strategically hung around the
Bivi to catch rainwater for consumption. When not underground the members
congregate here to discuss the exploration and organise future trips. Because
the depths at which we cave cave are so large, there is an underground camp at
-550m with a capacity of four so that teams can rest between pushing trips.

The initial plan was to introduce us newbies gently to the huge depth of the
Slovenian caves with several trips preparing us for underground camp, although
this did not entirely work out. My first trip was to Gardener's world, also
known as Vrtnarija, one of the two main cave systems in Migovec. Having heard
of how Slovene caves differed enormously to British caves I didn't really know
what to expect and found the prospect of entering this alien territory a little
daunting. The first thing I noticed was how sharp and jagged the rocks were so
that everything from my oversuit to cowstails got caught, frustratingly
hindering movement. This wasn't too much of a problem though, as the cave is
pleasantly wide and tall with very few passages requiring crawling or
squeezing. All of the hard work is in the large proportion of Single Rope
Technique needed. On my first trip we went to -130m, the depth of a good-sized
British cave, and it was lovely and very much in my comfort zone. We had,
however, only been down small to medium sized pitches; from where we had turned
round, at the top of a 60m pitch, things further on looked considerably more
intimidating.

After a day off from caving due to mad blisters, the opportunity arose to help
set up underground camp. I was initially a bit weirded out by this idea as I
came to Slovenia with the eventual goal to make it to underground camp and now
I was going to do what I came to do in 3 days time! However I am not one to say
no to a challenge so hastily made my way underground before I could change my
mind. A team consisting of another eager fresher, two experienced cavers and
myself made our way down the largest pitches I had ever seen, the largest being
Concorde (my favourite pitch), which at 90m high is big enough to fit a
Concorde in and has the most magnificent limestone formations. After a few
hours and considerably improving my descending technique we made it to the site
of underground camp. It was a long way down and after seeing how much rope we
had passed I was dreading prussicking up. I had never prussicked such a huge
distance before, how do I know if I can do it? We quickly set up camp by
pitching the tent, making the beds and unpacking food. We brought a MP3 player
and speakers down so that any fears or panic in my head were soon drowned out
by happy music and dancing. One thing I learnt from my month in Slovenia is
that David Bowie can make any dire situation a happy one. After having some
food and listening to the oh-so-homely Blackadder we went to sleep, 550m
beneath (almost) solid rock and completely disconnected from the world.

We were awoken by members who were on the night train (caving at night,
sleeping in the day) --- they had already been pushing and wanted our beds. We
reluctantly crawled out of our snugly sleeping bags into the 1oC cold and
quickly changed into our cold and wet caving gear. Next was the gruelling
ascent. Due to fear of exhausting myself I adopted a relaxed pace, taking a
whopping 8 hours to get out of the cave. Unfortunately for the member behind,
this meant waiting for me and led to attempts to speed me up such as force
feeding me chocolate and even singing. At the first glimpse of sunlight I
thought I was going to cry with happiness. It's strange that after a mere 24
hours without sunlight you miss it so much and all that physical effort just to
see it again makes it all the more magnificent. Unfortunately after this
feeling subsides you realise things were a lot more exciting underground and
that perhaps going back down is on top of the list of things to do.

The next few days were dedicated to treating blisters, nursing sore hands and
resting strained muscles. When I was suitably fit and there was a bed free in
underground camp the time arrived for my first pushing trip. We went down in a
team of three and shortly after entering the cave met another team who had
tales of their discovery of 'Wonderland', a passage with lots of promising
leads and pretty stals. Knowing these leads were ours to follow we eagerly
descended to underground camp. Some interesting incidents on the way down such
as my hair getting jammed in the descender and a scary slip ensured things
stayed exciting. I was glad to be back at underground camp, home sweet home.

Kicked out of bed at seven by the night train, I gingerly put on cold caving
gear, ate some fishy cheesy soupy smash and headed off to
'Wonderland'. Unfortunately this involved an encounter with the scariest
pitch I have ever and hopefully will ever come across. Initially named Leopard
due to the Leopard spot shaped mud splats on the walls, it soon became known
instead as Cheatah, due to that fact that a successful passage elicits a
feeling nothing less than one of cheating death. Whilst waiting for the pitch
to be rerigged in a safer fashion, a good deal of dancing was required to keep
warm. The caves after Cheatah are majestic. Large amphitheatre-like holes,
gloriously decorated passages and chambers full of large rocks to climb over,
wiggle between and slide under. The name Wonderland was well deserved.

However, as wondrous as it all was, exploring this only recently discovered
cave and wandering further and further away from camp really scared me.
Sleeping so deep underground, waking up and travelling even further into the
abyss sent me a little crazy. Normal life has never seemed so far away.
Thankfully a new, never explored pitch was quickly rigged and as the newbie I
was allowed to descend first into the virgin cave. Fear was soon replaced with
excitement, I would be the first person ever in the history of everything to
see and touch and just be in this part of the world. The pitch led to a nice
sized chamber with a stream way at the bottom. No waiting for the other two, I
scampered into the narrow passage, which became a pretty, winding stream way
with crystal clear water and white limestone. This ended with a pitch that
ended our pushing (but began someone elses). After agreeing on the name
'Serpentine', due to its snake-like meandering and association with the
Serpentine Lake, we began the arduous task of surveying the new cave. After
surveying we headed back to camp, tired and emotionally drained. Cheatah was no
more pleasant going up than coming down, the slippery mud made it frustratingly
tricky to reach the top. Glad to be back at camp.

After 6 hours of prussicking we made it for sunset and enjoyed the relaxed life
of the plateau. The next day the survey data was entered into the computer and
we saw our new passage in 3D and linked to the rest of the system. One week of
caving and already 1.5km of cave ha been discovered! Over the next few days the
rain came which kept the cavers underground and the rest of us huddled in the
bivi. These days were dedicated to games of chess and cards. After the rain had
ceased I had a little trip down system Migovec, the more thoroughly explored
system that it is hoped connects with Vrtnarija. We had a day dedicated to
scaling the nearby peaks, which provided some exhilarating climbing. We then
travelled down to Tolmin and enjoyed luxuries such as pizza and swims in the
emerald green water of the So\v{c}a river.

Showered and rejuvenated we headed back to the plateau, anxious to get back
underground. Due to the amount of interest from Slovenians and ICCC members
alike to get in on the action it wasn't till the final days of the caving
period that I got to return to camp. Two of us were to pack up camp and
hopefully get some pushing done in-between. 
Camp was not as friendly as in the early days with piles of litter, waiting to be carried out, tarnishing the once pure environment.
After a long sleep with no one on the night train to wake us
up we reluctantly got out into the cold and made our way to the pushing front.
Although we found no further leads I was given a tour of the majority of the
new findings this year. We saw some of the strangest formations such as several
spirals of mud that looked like a plug had been pulled beneath them. The huge
rifts and chambers are glorious and lots of fun to explore. We headed back to
camp to finish the pack up and slept at camp for the final time this year. Had
my last serving of fishy soupy cheesy smash (Thank god!) and went on my way.
Heading out we met the various teams sent down to carry the remaining bags who
made their presence known with singing heard from many pitches away.

The next few days were spent packing away the bivi, removing all evidence of
our presence in the national park. After saying goodbye to the plateau we
travelled to Tolmin and stayed in a member's flat. The next few days were spent
giving presentations to the enthusiastic locals about our exploration. A total
of 2.2km of new cave all below -550m left spirits on a high. A connection
between the two systems now looks ever more likely which, if found, would bring
it close to being the longest cave in Slovenia. On our final night we
celebrated our achievements with the JSPDT with traditional Slovenian music,
dancing and drinking.

Very hungover, we packed the minibus and reluctantly left Tolmin. This will be
an experience I will never forget. Not only has my caving vastly improved and
my thirst for caving increased but it was the first of many caving expeditions
I will be part of. The excitement of caving in new countries will always
enthuse me but I have the feeling that as many times I may explore the deep
caves of Tolminski Migovec we will always have unfinished business.


